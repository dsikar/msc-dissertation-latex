%----------------------------------------------------------------------------------------
%	ABSTRACT PAGE
%----------------------------------------------------------------------------------------

\begin{abstract}
% LC - 181 words - https://wordcounter.net/
\addchaptertocentry{\abstractname} % Add the abstract to the table of contents

%%% APPLICATION AND SUCCESS STORY OF CONVNETS 
%%% ConvNets has found success in diverse areas.
%%% Deep learning has found successful application in diverse areas
%%% including end-to-end self-driving cars, where a trained model is given an image and outputs values such as steering and throttle.

%%% THE PROBLEM WE ARE DEALING WITH - cONVNETS LACKING ROBUSTNESS E.G. ONE-PIXEL-ATTACKS, RANDOM LABELS (BENGIO ET AL)
However, there is a problem - noisy images

%%% WHAT WE DID - maybe leave the nitty gritty for later e.g. Unity 3D ?
This project investigates the effect of rainy images on the performance of end-to-end ConvNets applied to self-driving cars. using public domain datasets and synthetically generated by the Unity 3D gaming engine. network architectures. Gathering public datasets and generating datasets with the use of Unity 3D gaming engine. Rainy images were identified though the use of Amazon Mechanical Turk and also digitally processed to simulate rain.

%%% WHAT WE FOUND


%%% Results suggest that random noise does have a detrimental effect on network performance which may be mitigated by 

Convolutional Neural Networks (ConvNets) have become state-of-the-art in supervised image classification tasks, used in both \textit{pipelined} and \textit{end-to-end} architectures. the former relying on a fusion of image and sensor data the latter on images alone, to determine steering and movement output.  
However, ConvNets applied to image classification have been shown to lack robustness when, given noisy images, in some cases a single pixel change, are known to output incorrect results that for humans would have been obviously the same as the denoised version.
As deep learning models, be it ConvNets, Recurrent Neural Networks or Deep Reinforcement Learning Networks rely increasingly on computer vision for robotics in general, and for self-driving cars in particular, it is important to evaluate the effect of random noise on images used by such models.
As many car manufacturers and technology companies are actively developing self-driving cars, it is important to gauge the effect of noisy patterns such as rain on the performance of the self-driving car CNNs.
This study provides a performance comparison between a number of ConvNets architectures applied to autonomous driving, that can be used to inform design decisions.

%%% By training and testing resulting models with the Unity 3D game engine, we provide
  
\vspace{25mm} %5mm vertical space
  
  
\textbf{Keywords:} autonomous vehicles, convolutional neural networks, random image noise

\end{abstract}