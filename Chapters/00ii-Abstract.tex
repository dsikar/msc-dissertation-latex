%----------------------------------------------------------------------------------------
%	ABSTRACT PAGE
%----------------------------------------------------------------------------------------

\begin{abstract}
% LC - 181 words - https://wordcounter.net/
\addchaptertocentry{\abstractname} % Add the abstract to the table of contents

Convolutional Neural Networks are currently state-of-the-art in supervised image classification tasks, used in both \textit{pipelined} and \textit{end-to-end} architectures, the former relying on a fusion of image and sensor data the latter on images alone, to determine steering and movement output.  
However, CNNs have been shown to lack robustness when, given noisy images, in some cases a single pixel change, are known to output incorrect results that for humans would have been obviously the same as the denoised version.
As deep learning models, be it CNNs, RNNs or Deep RL rely increasingly on computer vision for robotics in general, and for self-driving cars in particular, it is important to evaluate the effect of random noise on images used by such models.
As many car manufacturers and technology companies are actively developing self-driving cars, it is important to gauge the effect of noisy patterns such as rain on the performance of the self-driving car CNNs.
This study provides a performance comparison between a number of CNN architectures applied to autonomous driving, that can be used to inform design decisions.

By training  and testing resulting models with the Unity 3D game engine, we provide

\textbf{Keywords:} autonomous vehicles, convolutional neural networks, rain, random image noise

\end{abstract}