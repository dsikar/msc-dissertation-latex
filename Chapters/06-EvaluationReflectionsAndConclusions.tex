
\chapter{Evaluation, Reflections and Conclusions}

\label{Eval} 

%This chapter should evaluate the project work as a whole. Here the original choice of objectives, the literature examined, the methods used, the planning, etc. are all reviewed to see what has been achieved by undertaking the project. There may be a summary of general conclusions drawn from the work done, highlighting the particular contribution of your project. You should also consider the implications of these conclusions. Discuss any proposals that you might make for further work, having discovered what you now know. It is also important to include a reflective section covering what you have learned from the project process. What would you do differently if you were to start again, knowing what you now know? Your report MUST include adequate Evaluation, Reflections and Conclusions to gain a passing grade.  

TBC - TODO. Some points to reflect:
\begin{itemize}
    \item[--] What this project demonstrated
    \item[--] What were the limitations
    \item[--] What have I learnt
    \item[--] What are the contributions
    \item[--] What would I have done differently - narrowed the scope for sure.
    \item[--] Here would be a good place, perhaps, to discuss "Understanding deep learning requires rethinking generalization", Zhang ICLR 2017, on deep models fitting random data (noise) perfectly. 
    \item[--] Recommendations for future work: although CNNs are not well understood, they are heavily used. A lot of effort has gone into creating models that are and could be harnessed for self-driving, such as ResNets and deeper architectures. These deeper models have been applied successfully to multi-class classification problems. Assuming the network design somehow is optimized for this type of task it would be interesting to transform regression problems into multi-class classification problems by quantizing (assigning a continuous value into a discrete value) and binning the outputs, subject to the quantized values having acceptable precision, in the case of a self-driving car, the minimum acceptable steering change, as the network output would become discrete and cease to be continuous as the network would not longer be a regression model.
    Perhaps such models could also produce usable results for computer vision applications.
    \item[--] Future work - add reflections to road. Experiment was limited to adding rain-like effects. With a better understanding of the game engine, it could be possible to add reflections to road, etc, and generate then another set of results and evaluation.
\end{itemize}
%The absence of good quality labelled datasets (maybe change this w.r.t. Mech Turk and ISLSVRC - "the presence of a good quality" labelled dataset engered a number of landmark computer vision CNN architectures, more obvious and less citations required) has been cited (TODO citation needed) as a hindrance to the development of good models (TODO need to insert in CONTEXT section "the term model hereafter, refers to a either a neural network architeture, or a trained model capable of making predictions given an input). It is ironic that the availability of good quality labelled models, such as Imagenet, led to the improvement in image classification models, which in turn may curb the need to use Mechanical Turk. This could perhaps form part of the wider discussion on automated workforce replacing human workforce and the social implications therein.  


%% Ethical issues - maybe leave to discussion and further work
%This also raises issues related to safety, liability, privacy, cybersecurity, and industry %risks \cite{Taeihagh_2018}
% Also, perhaps a discussion on social responsibility of developing systems that put people out of work

% to process more datasets, the maximum steering angle of vehicle must be known

% Reflection on the importance of naming conventions, over which some hours were spent
% to disambiguated runs. No sure what the best solution is here, but looks like 
% current scheme of YYYYMMDDHHMMSS_MODEL.h5 does not easily discriminate values as
% for instance RUN_XX_MODEL.h5 might have done. Even though models are repeated across
% runs, so that is also not ideal. Need to mention \label{app_res:62} and the 10 added 
% hidden units

% Reflect on the fact that feature maps for Alexnet could not be calculated according
% to Dumoulin and Visin, 2018 (Eq. 3.3). This should for part of a broader discussion
% stating the experience that models, be it Alexnet or NVIDIA, are not documented well
% enough to the point of being reproducible.

% Reflect on different hues generated by converting tcpflow packets into images. This differs from direct processing of Unity .jpg synthetically generated data.
%%%%%%%%%%%%%%%%%%%%%%%%%%%%%%%%%%%%%%%%%%%%%%%%%%%%%%%%%%%%%%%%%%%%%%%%%
% SUBMISSION CHECKLIST
% 1. This pdf
% 2. msc-dissertation-latex repository
% 3. SDSANDBOX repository
% 4. msc-data repository
% 5. Additional data - models, tcpflow logs, etc
%%%%%%%%%%%%%%%%%%%%%%%%%%%%%%%%%%%%%%%%%%%%%%%%%%%%%%%%%%%%%%%%%%%%%%%%%