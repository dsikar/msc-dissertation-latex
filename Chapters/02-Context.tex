
\chapter{Context}

\label{Context} 

This chapter explains the current state of your topic, in practice and theory. This is the state of the world which you intend to improve, and the state of knowledge on top of which you build your advances and from which you learn knowledge to apply and constraints on your work. So, you will report and analyse what is known about a certain topic, as reported in reference literature and published scientific literature; if you are developing a product, you will need to report about comparable or competing products over which you intend to improve or from which you will obtain ideas; you may need to describe legal or societal situation within which your work takes place; etc.  
  
It is important to demonstrate scholarship, i.e. the ability to read about a subject area in a range of sources, assimilate the material and then discuss it intelligently.  
  
You should demonstrate that you understand what you have read by providing some analysis or commentary in view of the goals of your project: it is not enough simply to provide summaries of what you have read. References should be cited following the Harvard Referencing Style. You must also explain, both in this chapter and, as appropriate, in others, how the results of the studies to which you make reference inform your project work. To gain a passing grade, your report MUST demonstrate adequate engagement with academic literature and any other sources necessary for the work to be well informed.  

Digital images are stored as a square multidimensional matrix. A 100x100 colour images (with values from Red, Green and Blue) is represented as a n x m x c matrix, where n is the height, m is the width and c is the depth (number of colour channels of the image. Black and white images have c = 1, where the pixel value is either on or off (TBC). Greyscale images.  

Maybe a bit of history on SVM, HOG, SURF, etc.  

Maybe a bit on how synthetic datasets are used in other domains e.g. https://arxiv.org/pdf/1910.02550.pdf "ClearGrasp:
3D Shape Estimation of Transparent Objects for Manipulation" use of Blender. See section "Learning from synthetic data"

%-----------------------------------
%	BACKGROUND
%-----------------------------------

\section{ImageNet Challenge}

The  ImageNet Large-Scale Visual Recognition Challenge (ILSVRC) (Russakovsky et al., 2014) played an important part in the development of deep neural networks for image recognition. With the exception of DriveNet (NVIDIA end-to-end self-driving ConvNet architecture), all remaining four (AlexNet, GoogleLeNet, VGG and ResNet) were winning entries in the ILSVRC.  

From the 

\section{Deep Learning applied to autonomous vehicles}

\subsection{Modular pipeline}

\subsection{End to end learning}



\lipsum[1]

%-----------------------------------
%	DATASETS
%-----------------------------------
\section{Datasets}

TODOS

\begin{itemize}
    \item Itemize / create table of datasets - see surveys
    \item Discuss importance of public datasets
    \item References
\end{itemize}

Advances in computer vision brought about by the Large Scale 

\lipsum[2]
