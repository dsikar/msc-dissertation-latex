
\chapter{Results}

\label{Results} 

This chapter presents the outputs that you produced, by applying the methods that you have selected, including e.g. analysis, design, prototyping, experimental work, evaluation, etc.  
  
How you report these results will depend on the nature of the work. It may be helpful to divide them into basic data (e.g., for a project that developed a software product, requirements specification, test data, etc.) and analysis of the data (e.g. statistical analyses, evaluation analyses, etc.). Remember that you are informing the reader of what you have produced and found and emphasising the interesting parts, so summarising at the end of each major section is useful.  
  
It is usually very helpful for the readers to include graphics and diagrams, for instance to clarify software design or requirements, identify key trends and relationships in empirical data, etc. If you do so, be sure to refer to these figures in the text and use them as evidence to support what you are explaining or arguing; and be sure that your figures are well designed and clearly presented – do not just use default settings of the software you are using in producing them.  
  
It is essential that you identify clearly what you accomplished or produced yourself, as opposed to what existed before you started your individual project or was provided by others. For instance, some projects build new software on top of an existing code base, add new data to an existing body of data, or are executed by a student as a member of a team. It is essential to indicate what parts of the activities and results which you report are your own work. If this is left unclear, the markers are instructed not to give credit for work that they cannot attribute to you. Ambiguity would attract penalties for poor academic practice, with delays caused by any investigation (deception would be treated as academic misconduct, of course, which may lead to expulsion).

%-----------------------------------
%	Network training
%-----------------------------------

\section{Network training}

% write up, label runs and make reference

Starting with the NVIDIA baseline, a number of hyperparameters were trialed. The initial setup failed to generate usable models. 
The table below presents training results for best trained models.

\section{Simulated self-driving car}

Using the baseline neural network architecture as described in 
Models trained with no image pre-processing, did not perform well, leading to cars driving off the road, as shown in Fig.  sequence.

% data gathered on Robot Racing League track
We gathered 10 laps of data on the Robot Racing League track, with maximum speed set to 2.1, proportional control set to 16 and differential set to 77. Maximum steer was set to 25 (degrees). Corresponding to 12778 .jpg image files and the same number of  .json files, containing corresponding throttle and steering angle values recorded at the moment image was saved by simulator. This can be seen in the calls to Update() and SaveCamSensor functions in  
\begin{verbatim}
./Assets/Scripts/Logger.cs
\end{verbatim}

