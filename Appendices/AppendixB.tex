\chapter{Network architectures} % Main appendix title

\label{AppendixB} % For referencing this appendix elsewhere, use \ref{AppendixX}

\section{NVIDIA baseline architecture}

This is the starting point after the work of \cite{bojarski2016end}

The network architecture is as shown, with a total of 817,028 trainable parameters.
\begin{verbatim}
import models
mymodel = models.get_nvidia_model1(2)
models.show_model_summary(mymodel)
Model: "model_3"
_________________________________________________________________
Layer (type)                 Output Shape              Param #   
=================================================================
img_in (InputLayer)          [(None, 120, 160, 3)]     0         
_________________________________________________________________
lambda_3 (Lambda)            (None, 120, 160, 3)       0         
_________________________________________________________________
conv2d_1 (Conv2D)            (None, 58, 78, 24)        1824      
_________________________________________________________________
dropout_15 (Dropout)         (None, 58, 78, 24)        0         
_________________________________________________________________
conv2d_2 (Conv2D)            (None, 27, 37, 32)        19232     
_________________________________________________________________
dropout_16 (Dropout)         (None, 27, 37, 32)        0         
_________________________________________________________________
conv2d_3 (Conv2D)            (None, 12, 17, 64)        51264     
_________________________________________________________________
dropout_17 (Dropout)         (None, 12, 17, 64)        0         
_________________________________________________________________
conv2d_4 (Conv2D)            (None, 10, 15, 64)        36928     
_________________________________________________________________
dropout_18 (Dropout)         (None, 10, 15, 64)        0         
_________________________________________________________________
conv2d_5 (Conv2D)            (None, 8, 13, 64)         36928     
_________________________________________________________________
dropout_19 (Dropout)         (None, 8, 13, 64)         0         
_________________________________________________________________
flattened (Flatten)          (None, 6656)              0         
_________________________________________________________________
dense_6 (Dense)              (None, 100)               665700    
_________________________________________________________________
dense_7 (Dense)              (None, 50)                5050      
_________________________________________________________________
steering_throttle (Dense)    (None, 2)                 102       
=================================================================
Total params: 817,028
Trainable params: 817,028
Non-trainable params: 0
_________________________________________________________________
[(None, 120, 160, 3)]
(None, 120, 160, 3)
(None, 58, 78, 24)
(None, 58, 78, 24)
(None, 27, 37, 32)
(None, 27, 37, 32)
(None, 12, 17, 64)
(None, 12, 17, 64)
(None, 10, 15, 64)
(None, 10, 15, 64)
(None, 8, 13, 64)
(None, 8, 13, 64)
(None, 6656)
(None, 100)
(None, 50)
(None, 2)    
\end{verbatim}

\section{Training and testing networks}
A network is trained by running the script:
\begin{verbatim}
$ python --model=nvidia1
    --outdir=../trained_models
    --epochs=1
    --inputs='../dataset/unity/roboleague/log/*.jpg'
    --aug=false
    --crop=false    
\end{verbatim}
This will generate a model, saved in the .h5 format, a python dictionary object with training and testing accuracy and loss values, and an accuracy and loss plot with information additional labels to help identify the trained model. Finally, a log file is saved to disk containing model name, training time and last recorded loss and accuracy values for training and testing datasets.

run experiment on vanilla code (no augmentation)
Record cash crash and deviation from actual to predicted steering angles.